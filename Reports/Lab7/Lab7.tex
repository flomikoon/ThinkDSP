\documentclass[a4paper,12pt]{report}

\usepackage{cmap}
\usepackage[T2A]{fontenc}
\usepackage[utf8]{inputenc}
\usepackage[english,russian]{babel}
\usepackage{listings}
\usepackage{amsmath}
\usepackage{float}
\usepackage{csquotes}
\usepackage{mathtools}
\usepackage{hyphenat}
\usepackage{amsfonts}
\usepackage{upgreek}

\usepackage{xcolor}
\usepackage{hyperref}

\usepackage{graphicx}
\graphicspath{ {./images/} }

\definecolor{dkgreen}{rgb}{0,0.6,0}
\definecolor{gray}{rgb}{0.5,0.5,0.5}
\definecolor{mauve}{rgb}{0.58,0,0.82}

\lstset{
    language=Python,                
    basicstyle=\small\sffamily, 
    numbers=left,               
    numberstyle=\tiny,           
    stepnumber=1,                 
    numbersep=5pt,            
    aboveskip=3mm,
    belowskip=3mm,
    showstringspaces=false,
    columns=flexible,
    captionpos=b, 
    basicstyle={\small\ttfamily},
    numbers=left,
    numberstyle=\tiny\color{gray},
    keywordstyle=\color{blue},
    commentstyle=\color{mauve},
    stringstyle=\color{dkgreen},
    breaklines=true,
    breakatwhitespace=true,
    tabsize=3
}

\title{Лабораторная работа №7\\Дискретное преобразование Фурье}
\author{Крынский Павел}
\date{\today}

\begin{document}

\maketitle
\tableofcontents
\lstlistoflistings

\maketitle

\chapter{Упражнение 7.1}

В данном упражнении нас просят открыть \texttt{chap07.ipynb}, прочитать пояснения и запустить примеры. Поэтому я просто изучил все примеры и комментарии к ним.


\chapter{Упражнение 7.2}

В дааном упражнении предлагается иследовать и релизовать бытрое преобразование Фурье. Реализация заключается в делении массива значений некоторого сигнала на подмассивы с четными и не четными элементами, вычислении ДПФ для них.

Возьмём небольшой реальный сигнал и вычислим его БПФ:

\begin{lstlisting}[caption=Вычисление БПФ при помощи \texttt{np.fft.fft}]
ys = [-0.2, 0.3, 0.5, -0.3]
hs = np.fft.fft(ys)
print(hs)
\end{lstlisting}

Результаты применения к нему быстрого преобразования: [ 0.3+0.j  -0.7-0.6j  0.3+0.j  -0.7+0.6j].

Напишем функцию для БПФ:

\begin{lstlisting}[caption=Функция \texttt{dft}]
def dft(ys):
    N = len(ys)
    ts = np.arange(N) / N
    freqs = np.arange(N)
    args = np.outer(ts, freqs)
    M = np.exp(1j * PI2 * args)
    amps = M.conj().transpose().dot(ys)
    return amps
\end{lstlisting}

Удостоверимся, что эта реализация даёт тот же результат, что и \texttt{np.fft.fft}.

\begin{lstlisting}[caption=Сравнение реализаций]
hs2 = dft(ys)
sum(abs(hs - hs2))
\end{lstlisting}

Все верно и результаты сходятся \texttt{5.552192056508194e-16}.  

Чтобы сделать рекурсивное БПФ, я начну с версии, которая разбивает входной массив и использует \texttt{np.fft.fft} для вычисления БПФ половин.

\begin{lstlisting}[caption=Функция \texttt{fft\_norec}]
def fft_norec(ys):
    N = len(ys)
    He = np.fft.fft(ys[::2])
    Ho = np.fft.fft(ys[1::2])
    
    ns = np.arange(N)
    W = np.exp(-1j * PI2 * ns / N)
    
    return np.tile(He, 2) + W * np.tile(Ho, 2)
\end{lstlisting}

Прверяем результат.

\begin{lstlisting}[caption=Сравнение реализаций]
hs3 = fft_norec(ys)
sum(abs(hs - hs3))
\end{lstlisting}

И получим ошибку равную \texttt{1.1102230246251565e-16}

Теперь заменим библиотечную функцию \texttt{np.fft.fft} на рекурсивными вызовы и добавим базовый случай:

\begin{lstlisting}[caption=Функция \texttt{fft}]
def fft(ys):
    N = len(ys)
    if N == 1:
        return ys
    
    He = fft(ys[::2])
    Ho = fft(ys[1::2])
    
    ns = np.arange(N)
    W = np.exp(-1j * PI2 * ns / N)
    
    return np.tile(He, 2) + W * np.tile(Ho, 2)
\end{lstlisting}

Результаты снова совпадают:

\begin{lstlisting}[caption=Сравнение реализаций]
hs4 = fft(ys)
sum(abs(hs - hs4))
\end{lstlisting}

Разница между результатами \texttt{3.1401849173675503e-16}.
Эта реализация БПФ требует времени и пространства, пропорционального $nlogn$ и это тратит время на создание и копирование массивов. Его можно улучшить, чтобы он работал «на месте», но в этом случае он не требует дополнительного места и тратит меньше времени на накладные расходы.

\chapter{Выводы}

Во время выполнения лабораторной работы получены навыки работы с комплексными экспонентами, а также с дискретным преобразованием Фурье.

\end{document}